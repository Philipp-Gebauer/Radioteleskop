% !TEX root = main.tex
\section{Grundlagen}
Wie vorab besprochen wurden die folgenden Themen bereits während des Vorkolloquiums diskutiert und werden daher nicht im Detail ausgeführt:
\begin{itemize}
    \item Hyperfeinstruktur im Wasserstoffatom
    \begin{itemize}
        \item[→] Wechselwirkung von Kern- und Elektronenspin, Entstehung der \SI{21}{\centi \metre}-Linie, deren Lebensdauer, Einfluss der \textsc{Doppler}-Verschiebung
    \end{itemize}
    \item Rotationsmodelle (differentiell, Kepler und starrer Körper) und deren Vorliegen in Milchstraße oder unserem Sonnensystem
    \item Koordinatensysteme (Galaktisch, Horizontal und Äquatorial)
    \begin{itemize}
        \item[→] Wann sind welche Bereiche der Milchstraße am besten zu beobachten, sowie das Überprüfen via \textsc{Stellarium} 
    \end{itemize}
    \item Verständnis für die Größenordnungen
    \begin{itemize}
        \item[→] Milchstraße vs. Sonnensystem (Parsec [\si{\parsec}] und Lichtjahr [\si{\lightyear}])
    \end{itemize}
\end{itemize}