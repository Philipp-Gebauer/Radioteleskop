% !TEX root = main.tex
\section{Einleitung}
Die Astrophysik beschreibt ein großes Feld physikalsicher Forschungsmöglichkeiten.
Dabei liegen stets kaum erfassbaren Größenordnungen zugrunde.
So reiste beispielsweise die in diesem Versuch gemessene Strahlung mehrere Jahre bei Lichtgeschwindigkeit durch das Universum um schließlich in Onsala gemessen werden zu können.
Dies veranschaulicht wie unbegreifbar groß die Weiten und damit auch die Forschungsmöglichkeiten dort sind.
Allein unser Sonnensystem liegt ca. 25 Lichtjahre vom Zentrum der Milchstraße entfernt.
Dennoch ist es mit recht einfachen Mitteln möglich Teile der Milchstraße zu kartografieren, wie der vorliegende Bericht dokumentiert.
\newline