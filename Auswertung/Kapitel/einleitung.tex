% !TEX root = main.tex
\section{Einleitung}
Die Astrophysik beschreibt ein großes Feld physikalischer Forschungsmöglichkeiten. Mithilfe des Wissens über die Stabilität und den Ursprung des Universums lassen sich womöglich viele neue Bereiche der Physik, vor allem im Bereich der Energieforschung, ermöglichen.
In der Astrophysik liegen stets kaum erfassbare Größenordnungen zugrunde.
So benötigte beispielsweise die in diesem Versuch gemessene Strahlung mehrere Jahre bei Lichtgeschwindigkeit durch das Universum, um schließlich mit einem Radioteleskop in Onsala gemessen werden zu können.
Dies veranschaulicht, wie unbegreifbar groß die Weiten und damit auch die Forschungsmöglichkeiten dort sind.
Allein unser Sonnensystem liegt ca. 25 Lichtjahre vom Zentrum der Milchstraße entfernt.
Dennoch ist es mit recht einfachen Mitteln möglich, Teile der Milchstraße zu kartografieren, wie der vorliegende Bericht dokumentiert.
\newline