% !TEX root = main.tex
\title{Radioteleskopie}
\subtitle{Physikalisches Fortgeschrittenenpraktikum an der Universität Konstanz}
\author{Autoren: Philipp Gebauer und Simon Keegan \\ \large{Tutor: Stefan Schupp}}
\date{Versuch durchgeführt am 13.05.2020}
\maketitle
\vspace{2.5 cm}
\begin{abstract}
    \vspace{1cm}
    \noindent \textbf{Abstract:}
    \begin{singlespace}
    This experiment aims to map a wide range of astronomical objects in the Milky Way. Therefore an antenna in Onsala, Sweden was used remotely.

    As a first step the main characteristica of this antenna were examined.\\
    By measuring the total power of the sun and taking a grid image of the sun's radiation profile the sun was legitimized as a point source. \\
    Based on the assumption that the \textsc{Salsa}-telescope can be idealized as an aperture, mathematically described by a $\sinc$-function, a \textsc{gaussian} fit was applied to the data. Thus FWHM refered to as angular resolution was quantified as $\SI{6.6 \pm 1.1}{\degree}$. Inferred from that the diameter of the telescope was calculated as $\SI{2.23 \pm 0.37}{\metre}$ in good accordance to the real value of $\SI{2.3}{\metre}$ \cite{Usermanual}. At last a degree of intrinsic inaccuracy of the telescope was obtained and measured as $\SI{0.976 \pm 0.034}{\degree}$ from the \textsc{gaussian} fit.\newline
    The measurements at $\SI{1410}{\mega \hertz}$ (H1 line; $\lambda = \si{21}{cm}$) results in a constant velocity of all objects in the Milky Way of $\SI{210.9 \pm 3.1}{\frac{km}{s}}$ which is really close to the literature ($\SI{220}{\frac{km}{s}}$ \cite{LSR}). The verification of the constant velocity is an indirect proof of dark matter.\newline
    The mapping of the Milky Way results in good values wich are consistent with the literature. At least three different spiral arms of the galaxy are visible with the calculated data.
    \end{singlespace}
\end{abstract}
\thispagestyle{empty}
\newpage