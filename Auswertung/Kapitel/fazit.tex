% !TEX root = main.tex
\section{Fazit}
In Bezug auf die Beobachtung der Sonne kann ein durchweg positives Fazit gezogen werden.
Einige Annahmen -- die Sonne als Punktquelle oder das Teleskop als Lochblende aufzufassen -- und daran anknüpfende Folgerungen wie das Erhalten der $\sinc$-Funktion oder das Berechnen des Auflösungsvermögens mittels FWHM wurden durch die guten Ergebnisse und anschaulichen Grafiken legitimiert.
Bis auf die Positioniergenauigkeit -- $\SI{0.976 \pm 0.034}{\degree}$ gegenüber $\SI{0.5}{\degree}$ -- des Teleskops liegen alle ermittelten Charakteristika des Teleskops auf Grundlage der Unsicherheiten in guter Übereinstimmung mit den Werten der Projekt-Dokumentation \cite{Usermanual}.
Auch die Positioniergenauigkeit liegt in derselben Größenordnung und widerspricht somit nicht gänzlich der Erwartung.
Das Auflösungsvermögen ($\SI{6.6 \pm 1.1}{\degree}$ zu $\SI{6}{\degree}$ \cite{Usermanual}) und das Rückrechnen auf den Teleskopdurchmesser ($\SI{2.23 \pm 0.37}{\metre}$ zu $\SI{2.3}{\metre}$ \cite{Usermanual}) zeigen wie erwähnt gute Übereinstimmungen.\newline
Auch die Messungen und Berechnungen der Milchstraße lieferen sehr gute Werte, welche stets in guter Übereinstimmung mit der Literatursind. Ausschlaggebende Erfolge dieses Versuchsteils ist zum einen die nahezu konstante Geschwindigkeit der Körper in der Milchstraße von $\SI{210.9 \pm 3.1}{\frac{km}{s}}$ (Literaturwert $\SI{220}{\frac{km}{s}}$ \cite{LSR}), was deutlich auf eine unbekannte Energie und Materie im Universum hindeutet, und die sehr präzise Kartografie der Milchstraße. Hier sind die Seitenarme \textit{Cygnus-Arm}, \textit{Perseus} und \textit{Orion} anhand eines Vergleichs mit der Literatur bestimmbar.
Somit war ein hinreichend präzises Vermessen der Milchstraße gewährleistet.