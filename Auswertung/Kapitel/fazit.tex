% !TEX root = main.tex
\section{Fazit}
Abschließend kann festgehalten werden, ...
%Simon zusatz

In Bezug auf die Beobachtung der Sonne kann ein durchweg positives Fazit gezogen werden.
Einige Annahmen -- die Sonne als Punktquelle oder das Teleskop als Lochblende aufzufassen -- und daran anknüpfende Folgerungen wie das Erhalten der $\sinc$-Funktion oder das berechnen des Auflösungsvermögens mittels FWHM wurden durch die guten Ergebnisse und anschaulichen Grafiken legitimiert.
Bis auf die Positioniergenauigkeit -- $\SI{0.976 \pm 0.034}{\degree}$ gegenüber $\SI{0.5}{\degree}$ -- des Teleskops liegen alle ermittelten Charaktersitika des Teleskops auf Grundlage der Unsicherheiten in guter Übereinstimmungn mit den Werten der Projekt Dokumentation \cite{Usermanual}.
Auch die Positioniergenauigkeit liegt in der selben Größenordnung und widerspricht somit nicht gänzlich der Erwartung.
Auflösungsvermögen ($\SI{6.6 \pm 1.1}{\degree}$ zu $\SI{6}{\degree}$ \cite{Usermanual}) und das Rückrechnen auf den Teleskopdurchmesser ($\SI{2.23 \pm 0.37}{\metre}$ zu $\SI{2.3}{\metre}$ \cite{Usermanual}) zeigen wie erwähnt gute Übereinstimmungen.
Somit war ein hinreichend präzises Vermessen der Milchstraße gewährleiste.

%Simon zusatz
