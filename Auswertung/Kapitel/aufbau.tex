% !TEX root = main.tex
\section{Aufbau und Durchführung des Versuchs}
\label{sec:Aufbau}
Für die Messungen der Frequenz- bzw. Geschwindigkeitsspektren astronomischer Objekte in der Milchstraßenebene wurde das Radioteleskop ,,Brage`` des \textsc{Onsala Space Observatory} in Schweden genutzt.
Dazu wurde das Teleskop der \textsc{Salsa}-Onsala-Einrichtung per Remote-Steuerung bedient und dabei interne Software der Forschungsanstalt genutzt.\\
Gemäß der Dokumentation \cite{Usermanual} und \cite{AntennaResp} beträgt der Durchmesser des Teleskop $\SI{2.3}{\metre}$. Die Track-Genauigkeit wird auf $\SI{0.5}{\degree}$ und das Auflösungsvermögen auf $\SI{6}{\degree}$ bei $\SI{1420}{\mega \hertz}$ beziffert.
Eine Bandbreite von $\SI{2}{\mega \hertz}$ mit 256 Kanälen wurde genutzt, dabei beträgt die Frequenzauflösung pro Kanal $\SI{7.8}{\kilo \hertz}$.
Da \textsc{Salsa} noch nicht Flux kalibriert ist werden Insensitäten stets in ,,arbitrary units``' angegeben und bei der Messung von Antennen Temperaturen sind nur Betrachtungen relativer Werte aussagekräftig.
Zudem sollte zum Erhalt guter Spektren stets bereiche über $\SI{15}{\degree}$ horizontaler Höhe und Messdauern von über $\SI{20}{\second}$ beachtet werden.\\%Zur Reduktion des Rauschens RFI und evtl LSR? oder unnötig?
%Welche Bereiche waren nicht betrachtbar?
Für die Vermessungen von Milchstraße und Sonne konnten verschiedene Frequenzen betrachtet werden ($\SI{1420.4}{\mega \hertz}$ respektive $\SI{1410}{\mega \hertz}$) sowie zwischen ,,Switched``- -- zur Minderung von Rauschen --  und ,,Signal``-Modus -- zur besseren Intensitätsmessung -- gewechselt werden.
Zudem konnten je nach Anforderung galaktische Koordinaten übergeben oder spezielle astronomische Objekte wie die Sonne direkt getract werden.
Auch azimutaler und Höhenoffset, sowie die Messdauer waren einstellbar.\\ 
\\ 
In Vorbereitung der Messungen wurde mittels des Programms \textsc{Stellarium} der am Messtag 13. Mai zur Messzeit zwischen 7 bis 13 Uhr vermessbare Bereich der Milchstraße ermittelt.
Am Versuchstag wurden dann zunächst die Areale der Milchstraße vermessen welche als erste den beobachtbaren Bereich verlassen.
Dabei wurden in Schritten von $\SI{5}{\degree}$ galaktischer Länge von $\SI{33}{\degree}$ bis $\SI{103}{\degree}$, jeweils bei $\SI{0}{\degree}$ galaktischer Breite und im ersten Quadranten ($\SI{0}{\degree}$ bis und $\SI{90}{\degree}$ galaktischer Länge), Spektren aufgezeichnet.
Anschließend wurden die Belichtungsdauern bei fester galaktischer Länge ($\SI{84}{\degree}$) und Breite ($\SI{0}{\degree}$) zwischen $\SI{1}{\second}$, $\SI{3}{\second}$, $\SI{10}{\second}$, $\SI{30}{\second}$, $\SI{100}{\second}$ und $\SI{300}{\second}$ variiert.
Dann wurde in Schritten von $\SI{10}{\degree}$ galaktischer Länge von $\SI{113}{\degree}$ bis $\SI{203}{\degree}$ jeweils bei $\SI{0}{\degree}$ galaktischer Breite Spektren aufgezeichnet.
Zuletzt wurden, bei möglichst hohem Sonnenstand, folglich möglichst zu lokaler Mittagszeit, Spektren der Sonne gewonnen.
Zum einen wurde ein Raster durch 25 Messungen generiert.
Dabei wurde eine relative Weite des Rasters von $\SI{5}{\degree}$ genutzt und durch azimutalen und Höhenoffset relativ zur Sonne das Raster erstellt.
Zum anderen wurde ein Kreuzscan der Sonne durchgeführt.
Dabei wurde in $\SI{2}{\degree}$-Schritten von $\SI{-16}{\degree}$ bis $\SI{16}{\degree}$ relativem Offset sowohl in azimut wie in altitude gemessen. Die jeweils ander Größe wurde auf $\SI{0}{\degree}$ Offset gesetzt.