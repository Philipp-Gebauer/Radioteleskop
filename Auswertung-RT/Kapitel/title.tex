% !TEX root = main.tex
\title{SALSA Radioteleskop Remote}
\subtitle{Physikalisches Fortgeschrittenenpraktikum an der Universität Konstanz}
\author{Autoren: Philipp Gebauer und Simon Keegan \\ \large{Tutor: Stefan Schupp}}
\date{Versuch durchgeführt am 13.05.2020}
\maketitle
\begin{abstract}
    \begin{center}
        \Large{\textsf{\textbf{Abstract}}}
    \end{center}
    \vspace{0.75 cm}
    \begin{singlespace}
    \noindent This experiment aims to map hydrogen clouds in the Milky Way and additionally to characterize the used radio telescope.
    Therefore, a radio telescope in Onsala, Sweden was used remotely.
    As a first step the main characteristica of the telescope were examined.\\
    By measuring the total power of the sun and taking a grid image of the sun's radiation profile the sun was legitimized as a point source. \\
    Based on the assumptions that the \textsc{Salsa}-telescope behaves as a circular aperture and that the sun can be idealized as a point source the expected defraction pattern is given by a $\sinc$-function, therefore a \textsc{gaussian} fit was applied to the data.
    Thus FWHM referred to as angular resolution was quantified as $\SI{6.6 \pm 1.1}{\degree}$.
    Inferred from that the diameter of the telescope was calculated as $\SI{2.23 \pm 0.37}{\metre}$ in good accordance to the real value of $\SI{2.3}{\metre}$ \cite{Usermanual}.
    At last, a degree of intrinsic inaccuracy of the telescope was obtained and measured as $\SI{0.976 \pm 0.034}{\degree}$ from the \textsc{gaussian} fit.\newline
    The measurements at $\SI{1420.4}{\mega \hertz}$ (H1 line; $\lambda = \SI{21}{\centi \metre}$) result in an almost constant velocity for all observed hydrogen clouds in the Milky Way of $\SI{210.9 \pm 3.1}{\frac{km}{s}}$, which is really close to the literature value ($\SI{220}{\frac{km}{s}}$ \cite{LSR}).
    The verification of the constant velocity is an indirect proof of dark matter.\newline
    The mapping of the Milky Way results in good values, which are consistent with literature. 
    At least three different spiral arms of the galaxy are visible with the calculated data.
    \vspace{0.75 cm}
     
    \noindent Beide Autoren haben zu jedem Abschnitt wesentliche Beiträge geleistet. Die Autoren bestätigen, dass sie die Ausarbeitung selbstständig verfasst haben und alle genutzten Quellen angegeben wurden.

\end{singlespace}
\end{abstract}

\thispagestyle{empty}
\newpage